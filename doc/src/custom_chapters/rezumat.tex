\thispagestyle{plain}
\begin{center}
\Large\textbf{\xthesistitle}
\end{center}
\parskip 0pt 
\begin{center}
Dan Cojocaru             
\end{center}
\begin{abstract}
Proiectul de licență presupune realizarea unei aplicații web de tip rețea de socializare ce are ca
scop crearea unei comunități virtuale. Aplicația numită „GoodCitizen” are rolul de a stimula spiritul 
civic al utilizatorilor din comunitățile reale/virtuale prin funcționalitățile unice puse la dispoziție. 
Funcționalitățile permit utilizatorilor să împărtășească evenimente și acțiuni din viața acestora 
ce au adus un beneficiu comunității. De exemplu participarea la acte de caritate, voluntariat sau orice
alt tip de binefacere. Aplicația dispune un sistem de premiere a utilizatorilor bazat pe interacțiunile
în comunitate. 

Aplicația web este o soluție \textit{Single Page Application}.
Aceasta este un tip de aplicație web care folosește o singură pagină
HTML, încarcă toate fișierele de tip CSS, JavaScript la primul apel și modifică dinamic
conținutul acestei pagini în funcție de acțiunile utilizatorului. 
Aplicația a fost implementată cu soft-cadrul Angular 2, folosind limbajul TypeScript. Acesta 
este un superset al limbajului JavaScript. În implementarea interfeței aplicației am folosit HTML5 
și SCSS(extensie CSS).

Datorită utilizării acestor tehnologii, aplicația capătă avantajul portabilității, astfel
utilizatorul poate accesa funcționalitățile oferite de pe calculatorul personal, sau de
pe dispozitive mobile, elementele din pagină apărând diferit în funcție de dimensiunile
ecranului.

Aplicația web consumă servicii de tip REST. Limbajul de programare ales pentru dezvoltarea 
serviciilor web(REST Web API) este Java. Pentru dezvoltarea aplicației am folosit soft-cadrul
Spring, versiunea 4. 

Web API-ul a fost implementat după arhitectura \textit{3-straturi}. Arhitectura \textit{3-straturi} 
presupune gruparea funcționalităților dintr-o aplicație în 
straturi. Cele 3 straturi sunt : persistență, logică și servicii. Principalele beneficii ale arhitecturii
pe straturi sunt: abstractizarea, izolarea, mentenabilitatea, performanța, reutilizabilitate și testabilitate.

Pentru stocarea datelor am utilizat o bază de date relațională. Sistemul de gestiune a bazelor de date ales 
este PostgreSQL, deoarece este unul dintre cele mai populare SGBD-uri open-source în acest moment.
\end{abstract}
Consider că am reușit să ating toate cerințele funcționale și nefuncționale pe care mi le-am
propus. În acest moment rețeaua de socializare este complet funcțională. Un utilizator poate 
interacționa cu alte persoane prin intermediul realizărilor globale oferit în mod implicit, sau
prin construirea de comunități online cu realizări proprii.

În viitor, trebuie investit mai mult timp în testarea aplicației.Este necesară scrierea de teste unitare și
teste de integrare pentru toate modulele aplicației. Totodată trebuiesc scrise teste end-to-end pentru 
interfața cu utilizatorul.  

Aplicația „GoodCitizen” are ca scop moral principal, construirea și îmbunătățirea spiritului civic în oameni.

Consider ca rețeaua de socializare ar avea succes în rândul tuturor categoriilor de vârsta, atât timp 
cât utilizatorul are puțin spirit civic. Totodată aplicația are rolul de a atrage și schimba perspectiva
celor care poate nu au încă dezvoltat un spirit civic. Aceștia pot lua modelul utilizatorilor deja existenți 
în rețeaua de socializare.

Rețeaua de socializare poate fi extinsă adăugând noi funcționalități. Am considerat că o socializare 
se extinde în funcție de cerințele utilizatorilor și din acest motiv am adăugat sistemul de feedback.
Se pot adăuga funcționalități cum ar fi: adăugarea de comentarii la o realizare, îmbunătățiri în 
sistemul de premiere adăugând ranguri utilizatorilor,etc. 

Arhitectura aplicației și modalitatea în care a fost dezvoltată permite schimbări și adăugări de noi 
funcționalități.

În concluzie, alegerea arhitecturii a fost bună indiferent de funcționalitățile pe care mi le-am 
propus la începutul acestui proiect.
\section{Document structure}
This document is structured into chapters (\cmdprint{chapter}), sections (\cmdprint{section}), 
subsections (\cmdprint{subsection}) and sub-subsections (\cmdprint{subsubsection}).
For example, to make a chapter with the heading Introduction one must write 
\verb?\chapter{Introduction}?.
All the chapters, sections and subsections formatted by the \cmdprint{tuiasithesis.sty} file are 
automatically numbered.

\section{Tables}

The essentials for creating tables can be found in \cite{url:latexTables}.
Table~\ref{table:tableExample} was created using Listing~\ref{lst:latexTable}.
\begin{table}
    \centering
    \caption{Table example}
    \label{table:tableExample}
    \begin{tabular}{l | *{3}{l}}
       \hline
        & col 1 & col 1 & col 1 \\
       \hline
       row 1 & some data & some data & some data \\
       row 2 & some data & some data & some data \\
       row 3 & some data & some data & some data \\
       row 4 & some data & some data & some data \\
       \hline
    \end{tabular}
\end{table}
 
\begin{lstlisting}[caption={Code for creating a LaTeX table},label={lst:latexTable}]
\begin{table}
    \centering
    \caption{Table example}
    \label{table:table_example}
    \begin{tabular}{l | *{3}{l}}
       \hline
        & col 1 & col 1 & col 1 \\
       \hline
       row 1 & some data & some data & some data \\
       row 2 & some data & some data & some data \\
       row 3 & some data & some data & some data \\
       row 4 & some data & some data & some data \\
       \hline
    \end{tabular}
\end{table}
\end{lstlisting}


\section{Lists}

There are several methods of making lists but the most widely used are simple unordered and ordered lists.
An unordered list is made using:
\verb?\begin{itemize}[<marker>] \item ... \end{itemize}?
and the usual markers are:
\begin{itemize}
  \item \cs{textbullet} \textbullet
  \item \cs{textendash} \textendash
  \item \cs{textasteriskcentered} \textasteriskcentered
  \item \cs{textperiodcentered} \textperiodcentered
\end{itemize}  
An ordered list is made using:
\verb?\begin{enumerate}[<style>] \item ... \end{enumerate}?.
\begin{quotation}
The
optional <style> argument can be used to specify the style used to typeset the item counter.
An occurrence of one of the special characters A, a, I, i or 1 in <style> specifies that the
counter will be typeset using uppercase letters (A), lowercase letters (a), uppercase Roman
numerals (I), lowercase Roman numerals (i), or arabic numerals (1).
\sourceatright{<from the memoir class manual, pg. 135>}
\end{quotation}

\section{Font styles and sizes}

\begin{table}
    \centering
    \caption{Font style commands (two methods)}
    \label{table:fontstyles}
    \begin{tabular}{l*{2}{l}}
       \hline
       \textup{Upright shape} & \verb?\textup{Upright shape}? & \verb?{\upshape Upright shape}?\\
       \textit{Italic shape} & \verb?\textit{Italic shape}? & \verb?{\itshape Italic shape}?\\
       \textsl{Slanted shape} & \verb?\textsl{Slanted shape}? & \verb?{\slshape Slanted shape}?\\
       \textsc{Small Caps shape} & \verb?\textsc{Small Caps shape}? & \verb?{\scshape Small Caps shape}?\\
       \textmd{Medium series} & \verb?\textmd{Medium series}? & \verb?{\mdseries Medium series}?\\
       \textbf{Bold series} & \verb?\textbf{Bold series}? & \verb?{\bfseries Bold series}?\\
       \textrm{Roman family} & \verb?\textrm{Roman family}? & \verb?{\rmfamily Roman family}?\\
       \textsf{Sans serif family} & \verb?\textsf{Sans serif family}? & \verb?{\sffamily Sans serif family}?\\
       \texttt{Typewriter family} & \verb?\texttt{Typewriter family}? & \verb?{\ttfamily Typewriter family}?\\
       \hline
    \end{tabular}
\end{table}

\begin{table}
    \centering
    \caption{Font size commands (are relative to the document font size)}
    \label{table:fontsizes}
    \begin{tabular}{l*{1}{l}}
       \hline
       \miniscule miniscule 7pt & \verb?\miniscule?\\
       \tiny tiny 8pt & \verb?\tiny?\\
       \scriptsize scriptsize 9pt & \verb?\scriptsize?\\
       \footnotesize footnotesize 10pt & \verb?\footnotesize?\\
       \small small 11pt & \verb?\small?\\
       \normalsize normalsize 12pt & \verb?\normalsize?\\
       \large large 14pt & \verb?\large?\\
       \Large Large 17pt & \verb?\Large?\\
       \LARGE LARGE 20pt & \verb?\LARGE?\\
       \huge huge 25pt & \verb?\huge?\\
       \Huge Huge 30pt & \verb?\Huge?\\
       \HUGE HUGE 36pt & \verb?\HUGE?\\
       \hline
    \end{tabular}
\end{table}

\section{Bibliography}
In order to cite a reference the \cmdprint{cite} command must be used.
For example \verb?\cite{IEEEexample:article_typical}? will produce the following citation
 \cite{IEEEexample:article_typical}. 
An example of citing references is something like\ldots in \cite{IEEEexample:bluebookbook} \cite{IEEEexample:conf_typical} \cite{IEEEexample:urlsty}
the authors present their work.
!!! Only the references cited in the text will be added to the bibliography at the end of the document.



\normalsize
